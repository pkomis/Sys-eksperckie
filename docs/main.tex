\documentclass{article}
\usepackage[T1]{fontenc}
% Language setting
% Replace `english' with e.g. `spanish' to change the document language
\usepackage[polish]{babel}
\usepackage{float}
% Set page size and margins
% Replace `letterpaper' with `a4paper' for UK/EU standard size
\usepackage[letterpaper,top=2cm,bottom=2cm,left=3cm,right=3cm,marginparwidth=1.75cm]{geometry}
\linespread{1.5}

% Useful packages
\usepackage{amsmath}
\usepackage{csquotes}
\usepackage{graphicx}
\usepackage[colorlinks=true, allcolors=blue]{hyperref}
\usepackage{biblatex}

\title{Opis projektu - Dietetyka, konfigurowanie posiłków, potraw}
\author{Kacper Leszczyński, Piotr Komisarczyk, Jan Kruszewski, Adrian Rudź}
\begin{document}
\maketitle

% DO PRZEANALIZOWANIA BO TO JEST WYGENEROWANE
% DO PRZEANALIZOWANIA BO TO JEST WYGENEROWANE
% DO PRZEANALIZOWANIA BO TO JEST WYGENEROWANE
% DO PRZEANALIZOWANIA BO TO JEST WYGENEROWANE
% DO PRZEANALIZOWANIA BO TO JEST WYGENEROWANE
\section*{1. Wprowadzenie}
Celem projektu jest stworzenie systemu ekspertowego wspomagającego konfigurowanie posiłków i potraw w zależności od preferencji oraz potrzeb użytkownika. Projekt zostanie zaimplementowany w języku Prolog, zgodnie z zasadami budowy modularnych systemów ekspertowych.

\section*{2. Charakterystyka systemów ekspertowych}
System ekspertowy (SE) to program komputerowy, który symuluje podejmowanie decyzji przez człowieka–eksperta. Wykorzystuje bazę wiedzy (zawierającą fakty i reguły) oraz mechanizm wnioskowania, aby odpowiadać na pytania i sugerować rozwiązania.

\subsection*{Cechy SE:}
\begin{itemize}
  \item oddzielenie wiedzy od mechanizmu sterującego,
  \item reprezentacja wiedzy za pomocą reguł: \texttt{jeżeli... to...},
  \item dialog z użytkownikiem,
  \item możliwość pracy z niepełną wiedzą,
  \item stosowanie logiki rozmytej lub wielowartościowej.
\end{itemize}

\section*{3. Opis dziedziny: Dietetyka}
System będzie działał w dziedzinie dietetyki. Celem jest wsparcie użytkownika w doborze odpowiednich potraw na podstawie:
\begin{itemize}
  \item celu dietetycznego (np. redukcja masy ciała, przyrost masy, zdrowie),
  \item preferencji smakowych,
  \item alergii,
  \item czasu przygotowania,
  \item pory dnia,
  \item dostępnych składników.
\end{itemize}

\section*{4. Architektura systemu}
Zgodnie z założeniami projektowymi system powinien być modularny i składać się z następujących komponentów:

\begin{itemize}
  \item \textbf{Interfejs użytkownika} – prowadzi dialog i prezentuje sugestie.
  \item \textbf{Moduł reguł minimalnych} – umożliwia identyfikację niezbędnych informacji;
  \item \textbf{Baza wiedzy} – zawiera fakty i reguły opisujące składniki, potrawy, cele dietetyczne;
  \item \textbf{Mechanizm wnioskowania} – łączy fakty i reguły w celu uzyskania wniosków;
\end{itemize}

\section*{5. Przykład reprezentacji wiedzy w Prologu}
\subsection*{Fakty:}
\begin{verbatim}
skladnik(jajko, bialko, 78).
skladnik(platki_owsiane, weglowodany, 150).
alergia(uzytkownik1, gluten).
cel(uzytkownik1, odchudzanie).
\end{verbatim}

\subsection*{Reguły:}
\begin{verbatim}
mozna_zjesc(Uzytkownik, Danie) :-
    nie_zawiera_alergenu(Danie, Uzytkownik),
    odpowiada_celowi_dietetycznemu(Danie, Uzytkownik),
    dostepne_skladniki(Danie).

proponuj_posilek(Uzytkownik, Posilek) :-
    mozna_zjesc(Uzytkownik, Posilek),
    pasuje_do_pory(Posilek, sniadanie).
\end{verbatim}

\section*{6. Przykład dialogu z użytkownikiem}
\begin{verbatim}
System: Jaki masz cel dietetyczny?
Użytkownik: Odchudzanie

System: Czy masz alergie?
Użytkownik: Gluten

System: Czy jesz mięso?
Użytkownik: Nie

System: Proponuję:
- Śniadanie: Owsianka z owocami
- Obiad: Kasza z warzywami
\end{verbatim}

% \section*{7. Wnioski i możliwości rozbudowy}
% Projektowany system ekspertowy pozwala wspierać użytkownika w zakresie dietetyki w sposób zbliżony do konsultacji z dietetykiem. W przyszłości system można rozbudować o:
% \begin{itemize}
%   \item ocenę sezonowości produktów,
%   \item bardziej zaawansowane wnioskowanie rozmyte,
%   \item uczenie systemu na podstawie preferencji użytkownika.
% \end{itemize}

\section*{7. Atrybuty}

\begin{table}[H]
\begin{tabular}{|c|c|}
\hline
Atrybut            & Wartość                                                  \\ \hline
Alergie            & nisko/średno/wysoko alergiczne                           \\ \hline
Kaloryczność       & niskokaloryczne / średnio kaloryczne / wysoko kaloryczne \\ \hline
Indeks glikemiczny & niski/średni/wysoki                                      \\ \hline
Typ żywnosci       & Warzywa/Owoce/Pochodzenia zwierzecego/Nabiał/Zboża       \\ \hline
Tłuszcze           & ilość                                                    \\ \hline
Węglowodany        & ilość                                                    \\ \hline
Białko             & ilość                                                    \\ \hline
Alergie            &                                                          \\ \hline
                   &                                                          \\ \hline
\end{tabular}
\end{table}

\end{document}
